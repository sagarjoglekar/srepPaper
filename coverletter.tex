%% start of file `template.tex'.
%% Copyright 2006-2013 Xavier Danaux (xdanaux@gmail.com).
%
% This work may be distributed and/or modified under the
% conditions of the LaTeX Project Public License version 1.3c,
% available at http://www.latex-project.org/lppl/.
%Version for spanish users, by dgarhdez

\documentclass[11pt,a4paper,roman]{moderncv}        % possible options include font size ('10pt', '11pt' and '12pt'), paper size ('a4paper', 'letterpaper', 'a5paper', 'legalpaper', 'executivepaper' and 'landscape') and font family ('sans' and 'roman')
%\usepackage[spanish,es-lcroman]{babel}


% moderncv themes
\moderncvstyle{classic}                            % style options are 'casual' (default), 'classic', 'oldstyle' and 'banking'
\moderncvcolor{green}                              % color options 'blue' (default), 'orange', 'green', 'red', 'purple', 'grey' and 'black'
%\renewcommand{\familydefault}{\sfdefault}         % to set the default font; use '\sfdefault' for the default sans serif font, '\rmdefault' for the default roman one, or any tex font name
%\nopagenumbers{}                                  % uncomment to suppress automatic page numbering for CVs longer than one page

% character encoding
\usepackage[utf8]{inputenc}                       % if you are not using xelatex ou lualatex, replace by the encoding you are using
%\usepackage{CJKutf8}                              % if you need to use CJK to typeset your resume in Chinese, Japanese or Korean

% adjust the page margins
\usepackage[scale=0.75]{geometry}
%\setlength{\hintscolumnwidth}{3cm}                % if you want to change the width of the column with the dates
%\setlength{\makecvtitlenamewidth}{10cm}           % for the 'classic' style, if you want to force the width allocated to your name and avoid line breaks. be careful though, the length is normally calculated to avoid any overlap with your personal info; use this at your own typographical risks...

% personal data
\name{Sagar Joglekar}{\\King's College London}
\title{Coverletter}                               % optional, remove / comment the line if not wanted
\address{66 Fairfield way,}{Stevenage, North Hertfordshire}{UK: SG1 6BF}% optional, remove / comment the line if not wanted; the "postcode city" and and "country" arguments can be omitted or provided empty
\phone[mobile]{+44 750-622-5616}                   % optional, remove / comment the line if not wanted
%\phone[fixed]{+2~(345)~678~901}                    % optional, remove / comment the line if not wanted
%\phone[fax]{+3~(456)~789~012}                      % optional, remove / comment the line if not wanted
\email{sagar.joglekar@kcl.ac.uk}                               % optional, remove / comment the line if not wanted
\homepage{www.sagarjoglekar.com}                         % optional, remove / comment the line if not wanted
%\extrainfo{additional information}                 % optional, remove / comment the line if not wanted
%\photo[64pt][0.4pt]{picture}                       % optional, remove / comment the line if not wanted; '64pt' is the height the picture must be resized to, 0.4pt is the thickness of the frame around it (put it to 0pt for no frame) and 'picture' is the name of the picture file
%\quote{Some quote}                                 % optional, remove / comment the line if not wanted

% to show numerical labels in the bibliography (default is to show no labels); only useful if you make citations in your resume
%\makeatletter
%\renewcommand*{\bibliographyitemlabel}{\@biblabel{\arabic{enumiv}}}
%\makeatother
%\renewcommand*{\bibliographyitemlabel}{[\arabic{enumiv}]}% CONSIDER REPLACING THE ABOVE BY THIS

% bibliography with mutiple entries
%\usepackage{multibib}
%\newcites{book,misc}{{Books},{Others}}
%----------------------------------------------------------------------------------
%            content
%----------------------------------------------------------------------------------
\begin{document}
    %-----       letter       ---------------------------------------------------------
    % recipient data
    \recipient{Request for consideration for publication in }{Nature Scientific reports}
    \date{\today}
    \opening{Dear Sir/Madam,}
    \closing{Best Regards,}
%    \enclosure[Enclosed]{CV}          % use an optional argument to use a string other than "Enclosure", or redefine \enclname
    \makelettertitle
    
    
   I am writing to submit our manuscript titled: ``\textbf{Analysing network structures of conversations in an online suicide support forum}'', for consideration to be published in Scientific Reports.
   
   The paper reports a large scale data driven study of online conversation structures and the developed metrics that distinguish a suicide support conversation from a general conversation. We propose a new framework to analyse the structure of conversations with a particular user under consideration using a new way of conducting triadic census.  We name this framework ``Anchored Triadic Motifs'', which we test on r/SuicideWatch conversations with the common Reddit conversations as a null model. We show that particular triadic motifs are statistically much more prevalent in graphs created from r/SuicideWatch forum when compared against a general conversation. These overexpressed motifs show resemblance with a victim/user centric conversation which is shown to be a hallmark of perceived social support. We discuss the implications of these results to the field of clinical psychology. 
   
   Through this work we propose a novel way of understanding conversation structures and the corresponding interactions between users, in order to distinguish between a supportive and a general conversation. 
   We believe that this work has wider implications, especially in the era of online hate and abuse. The features of support found in the structure of a supportive dialogue may help us design safer and supportive online spaces.
  
  
   Concerning the review process, we would like to suggest the following experts as reviewers:
   \begin{itemize}
        \item Meeyoung Cha ,Associate Professor , KAIST. [\url{https://ds.kaist.ac.kr/professor.html}, \textit{meeyoungcha@kaist.ac.kr}]
        \item Munmun De Choudhury, Assistant Professor in the School of Interactive Computing at Georgia Tech. She has published research in reddit and mental health. [\url{http://www.munmund.net/}, \textit{mchoudhu@cc.gatech.edu}]
        \item Glen Coppersmith, founder \& CEO at Qntfy. Dr Coppersmith conducts research on social media and several mental health issues, such as suicide prediction. [\url{https://qntfy.com/},\texttt{ glen@qntfy.io}]
        \item Professor Iain Buchan, Clinical Professor in Public Health Informatics at the University of Manchester[\url{https://www.bmh.manchester.ac.uk/about/people/leadership/iain-buchan/},\textit{iain.buchan@manchester.ac.uk}]
        \item Raul J Mondragon, Senior Lecturer Queen Mary University London [\url{http://eecs.qmul.ac.uk/profiles/mondragonraul.html},\textit{r.j.mondragon@qmul.ac.uk}] 
  
    \end{itemize}
  
   Thank you for considering our manuscript and we look forward to your response.
   
   \makeletterclosing
    
\end{document}


%% end of file `template.tex'.
