\section{Discussion}

The results show that there are several factors which distinguish a a SuicideWatch (SW) conversation from a comparative baseline of  conversations gathered from the front page (FP) of Reddit. Based purely on the \textit{ structure} of the conversations, we identified four clear differences between the macroscopic features of SW and FP: 
%corpora:  
The speed or \textit{urgency of response to the OP in SW is faster than for FP}, which is as would be intuitively expected of a subreddit set up as a place of support for “vulnerable OPs”.  
\rd{Just as in group therapy, it is individual clients and their larger relationship within therapy that is the agent of change \cite{yalom_theory_1995}, and the same is reflected in this peer support forum.   
Features of relational communication \cite{rogers_overview_1983} are that} \textit{SW shows more symmetry and reciprocity than FP}, and the \textit{OP is central to the communication}.  \rd{Studying the interlocking and reciprocal effects of each interactor on the other has been key to understanding “therapy as a system” in face-to-face therapeutic encounters also \cite{de_shazer_putting_1991}.
What is radically different from a clinical context is that the posters are not healthcare seeking and may be on SW precisely because they are seeking alternative support and there is no ‘professional’ facilitating the discussion.  The closest analogy in a live group setting is “fishbowls” (used in certain group counselling courses) \cite{keim_groupwork_2013} where there is an inner ring of discussants (the OP and other posters on the thread) who are observed by an outer-ring of observers (in SW, a parallel may be drawn with the moderators who manually examine comments and delete those that threaten or violate the thread’s specified codes
and ban trolls 
\cite{choudhury_language_2017}).}

%\textit{SW also shows a higher semantic alignment between posts and their replies than FP.} From a clinician perspective, this is most closely allied to reflective listening which is used widely in counselling and psychological therapies including motivational interviewing \cite{levensky_motivational_2007}. The principle here is rather than attempting to solve the problem or to persuade others towards a different way of thinking, the responder is communicating that they understand and acknowledge the OP’s thoughts and feelings and to encourage rather than close down further discussion. \ns{In future work, the psychologists in this team are aiming to examine this phenomenon more closely using content analysis and coding.}

\rd{In communication accommodation theory (CAT)\cite{coupland_introduction_1988}, which was developed for face-to-face conversations between two people (a dyad) but has now been extended to mediated dyadic discussions (e.g. on Twitter) without temporal immediacy \cite{lipinski-harten_comparison_2012}, the concept of accommodation has two opposite forms: convergence and divergence.  Convergence is mimicking of the conversational partner's style and divergence is avoidance of the style. This phenomenon may be reflected in SW by less branching or digression in the conversation thread compared to FP.}

At a mesoscopic level, the most striking features of the anchored triadic motifs which occur in statistically significant numbers (shown with a solid, non hatched background in Fig. \ref{fig:motifs}) are that none of them involve all three nodes, suggesting that dyadic communications (e.g., providing an answer to a question) are the primary focus.  Of course it would be extrapolating to assume this is supportive communication, and there would need to be further qualitative research into the \textit{content} of the threads that demonstrated these motifs.  Similarly when considering the anchored triadic motifs which are under expressed in both FP and SW datasets (grey hatched in Fig \ref{fig:motifs}), it is worth noting that although these are statistically rare in the current work, they could be worth exploring in other datasets.

Both of the anchored triadic motifs that are over expressed in the baseline FP conversations and by comparison are under expressed in SW (motifs that are shown in green in Fig. \ref{fig:motifs}), show non-conversational, non-reciprocal patterns of serial communications between respondents to an OP (021C-c) and unidirectional response to an OP from one respondent (012-b). In contrast, those over expressed in SW (shown in red) have two arrow heads pointing towards the apex nodes, suggesting that communication is directed towards one participant. Except for 021U-a and -b, the other motifs over expressed in SW all have at least one bidirectional conversation, reflecting the high levels of reciprocity in SW. 
%The solid grey boxed anchored motifs have a common theme of a ‘warning sign’ of a ‘loose node’ in three cases, suggesting low levels of supportive interaction. An exception to this is 111D-a, which would need further analysis.


Internet health forums have been studied in several instances and their utility has been shown to be of value in cases of chronic illnesses\cite{Joglekar2018}, addictions\cite{wood2009evaluation} and mental health issues \cite{gkotsis2017characterisation,de2013social}. However, most of these studies have focused on quantitatively analysing
the content discussed and the linguistic signatures of how these communities interact.
Here, we have instead focused on developing ways to quantitatively analyse the \emph{structure} of online communication, and study how and whether this structure reveals patterns of peer and community support.
To that end, this is the first attempt at finding topological discriminatory factors between supportive and generic conversations on social media forums. Our focus on structure rather than content means that our methods can potentially be extended to other languages more easily.

The public health implications of this work are that the distinctive supportive network structures and the content of their posts should be studied in more detail to investigate what works well and why.  This could help educate peer moderators to have a better overview of the subreddits they moderate and the ongoing conversation. Topological features could be used in addition to the community signals they already use, such as numbers of upvotes or downvotes, or referring to comments flagged by community members  \cite{choudhury_language_2017}.  Similarly, studying the less supportive motifs could lead to insights into \emph{why} certain interactions are unhelpful, and might allow automated detection of such interactions so that moderators are able to moderate such comments in a more timely fashion.  The results obtained could also be used as a selection strategy for purposefully sampling more supportive networks. We believe that the novel framework for macro and meso analysis of supportive online communities we present here can provide important directions for future research in this area.  